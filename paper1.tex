% New = "Delta Factor" due to line width
\documentclass[a4paper, 11pt]{ article}

\usepackage{graphicx}
\graphicspath{ {~/Desktop/FINALpix/} }

\title{Delta Factor - A New Authorization Factor}
\author{Alles Rebel, Thomas Perez - CalStateLA}
\begin{document}
\maketitle

%---------------------------------------------------------------------------------------------------------------------

\section{Abstract}
TBA (As per professor) %This is modus operandi bc as the paper evolves, the abstract changes invariably
\section{Introduction}
\setlength{\baselineskip}{1.5\baselineskip}
\indent

Today’s phishing toolkits used by savvy user attacks are state-of-the-art and have a profound effect on all levels of our work, shopping, web-surfing and much more. The effect it has on users can be quite profound. Attackers from Israel, Iran, Russia, US, North Korea, UK, Europe, and others do this according to the Washington Post, obviously some much more than others as a whole. DDoS attacks by bots are one of many types of attacks, but one that most of us have encountered, perhaps unwittingly, is the  man-in-the-middle attack, eg., MITM. They use phishing toolkits for spoofing innocent users or bystanders. The immediate handling of these types of attacks and urgency for mitigation, elimination or deterrence is obvious. 


There are methods to avoid and/or determine the existence of a MITM. Traditionally there are eg., Machine Learning methods, ie., the ubiquitous trust-based methods, 2FA, and more new ways, methods and tools for detection of MITM and more, however they aren't used nor easily available to the general public.
The current status in the field including the best performing tool is from the basis of our research, with is from the SUNY - Stonybrook paper, "Catching Transparent Phish: Analyzing and Detecting MITM  Phishing Toolkits". The name of the extremely efficient tool that they developed is called, "phoca".
More will be exposed on this later. In brief however, it is discovered that via COTP and of course a shared secret and 2FA for security, the APPLICATION layer CAN BE able to help us reveal a MITM if one exists, as opposed to the denial of that concept by Stonybrook, again, knowing that they did a tremendous amount of work. 

What \textbf{we} have done is \textit{dispute} this claim by the Stonybrook research group, emphasizing again, that the work that they did is very professional and thorough, these points of relevance will also be exposed and explained in great detail later. 

\noindent
The relevance and importance of COTP will also be explained.
Source Code Available: https://github.com/allesrebel/cotp

The organization of the report is as follows:

\noindent
TBA\\
TBA...

%---------------------------------------------------------------------------------------------------------------------

\section{Related Work}
Catching Transparent Phish: Analyzing and Detecting MITM Phishing Toolkits (2021, ACM Conference on Computer and Communications Security)
Brian Kondracki, Babak Amin Azad, Oleksii Starov, Nick Nikiforakis.\\

\noindent
Inferring the Presence of Reverse Proxies Through Timing Analysis
Alexander, Daniel R.\\

\noindent
VisualPhishNet: Zero-Day Phishing Website Detection by Visual Similarity
Sahar Abdelnabi, Katharina Krombholz, Mario Fritz\\

%---------------------------------------

\section{Design Principles}
MITM Phishing TOOLkitS:\\
PHOCA - by our SUNY - Stonybrook research group\\
Evilginx, Muraena, and Modlishka \\

% ppt frames 11 , 12 , 13 , 14 . 15
% Paragraph 1: a general overview of the techniques that we used

\noindent	% frame 			15
Design Goals:\\
• Methodology to work on any OS, Windows-10, OSX, Linux distro's,  and/or 

Android\\
• Ability to work on websites today\\
• Anyone anywhere, should be able to use this to verify a website\\
• Users should have the tool and ability to detect MITMs\\
\textbf{• Validate the server - at the application level} (with a shared secret)\\
• Making MITM significantly harder.\\

\noindent
Figure-1 below shows the typical simplified architecture of a MITM 
Apparently, the software/interface for the MITM is transparent. What can dilute the transparency are tools like phoca or Delta-factor, ie., our tool.\\

\noindent % frame 			3 = arch of MITM PIX (In professional typesetting, this is called a strut.)
% 		[  : )  pix ] 
% \usepackage{graphicx}  				Atop - as a preamble
% \graphicspath{ {~/Desktop/FINALpix/} } 		Atop - as a preamble 
\includegraphics[width=\textwidth]{/ARCH_TYP_MPTK.png}
Fig-1 Architecture of a MITM\\ 


% Paragraph 2: introduction of DT techniques ...

% Paragraph 6: introduction of RF techniques






%------------------------------------------------------------------------------%---------------------------------------
%------------------------------------------------------------------------------%---------------------------------------

\section{CITATIONS} % OUT-OF-ORDER intentional. Point number before "CITATIONS" auto-changed 
\noindent
Public Internet Standards (Various Authors)

\noindent
https://datatracker.ietf.org/doc/html/rfc4226 - Creating a HOTP

\noindent
https://datatracker.ietf.org/doc/html/rfc6238 - Create a TOTP 

\noindent
https://datatracker.ietf.org/doc/html/rfc4086 - Making High Entropy Secrets

\noindent
https://datatracker.ietf.org/doc/html/rfc2104 - HMAC Algorithm\\

\noindent
Test Website Live: http://allesrebel.com\\ 

\noindent
Reference implementations provided by papers + standardized code\\
TOTP - https://datatracker.ietf.org/doc/html/rfc6238
HOTP - https://datatracker.ietf.org/doc/html/rfc4226 
COTP - xxxxxxxxxxxxxxx TBA\\

\noindent
Collecting TCP Timings via Performance API\\
https://developer.mozilla.org/en-US/docs/Web/API/Performance\\
Collecting TLS Information via Security Info from Extensions JS API\\
https://developer.mozilla.org/en-US/docs/Mozilla/Add-ons/WebExtensions/API/webRequest/SecurityInfo\\

\noindent
Store Secret via Browser Extension Storage API\\
https://developer.mozilla.org/en-US/docs/Web/API/Storage\\

\noindent
Crypto APIs + Time APIs to generate COTP\\
https://developer.mozilla.org/en-US/docs/Web/API/Crypto\\


\noindent
JA3, https://github.com/salesforce/ja3\\

\noindent
https://coveryourtracks.eff.org/, Proceedings of the Privacy Enhancing Technologies Symposium


%---------------------------------------------------------------------------------------------------------------------











\end{document}


