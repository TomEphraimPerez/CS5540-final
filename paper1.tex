% New = "Delta Factor" due to line width
\documentclass[a4paper, 11pt]{article} 				%Short reports and scientific journals
% \documentclass{report}							% Books and thesis
\usepackage[utf8]{inputenc}						% 128++ UTF-8 charset -----------------------
\usepackage{graphicx}
\graphicspath{ {/Users/thomasperez/Desktop/finalpix/} }		%abs 
% \graphicspath{{../pdf/}{D:\ImagesforProjectLatex}}

\title{Delta Factor - A New Authorization Factor}
\author{Alles Rebel, Thomas Perez - CalStateLA}
\begin{document}
\maketitle



%---------------------------------------------------------------------------------------------------------------------
% Following exactly professor's Canvas example
\section{Abstract}
\setlength{\baselineskip}{1.5\baselineskip}
((TBA as per professor - however...))\\ %This is modus operandi bc as the paper evolves, the abstract changes invariably
\noindent
Man\_In\_The\_Middle-ware, (MITM) is a rapidly growing field of website spoofing that is a challenge for everyone, especially for MITM mitigation and deterrence  developers to keep up with. Many papers have been published that explain the nuances, methods and lengths that the attackers use for "personal gain". According to [1], fairly spectacular results in detecting a MITM attack has been shown. This was accomplished by the group gathering attack data, metrics of the attacks, and analysis for many months, to eventually develop a custom phishing toolkit they named "phoca". That said, the target of our research is two-fold. Firstly, we show improvement in the Stonybrook claim via our novel composite-type methodology using COTP.  Secondly, we claim and show that despite Stonybrook's claim, it is possible and provable that application level checks based on easily-acccessable and well known network level features along with a secret, can indeed determine with fine granularity, a MITM or !MITM. Ie., an efficient and novel server authentication factor has been developed.  



%---------------------------------------------------------------------------------------------------------------------
\section{Introduction}
\noindent
Today’s phishing toolkits used by savvy user attacks are state-of-the-art and have a profound effect on all levels of our work, shopping, web-surfing and much more. The effect it has on users can be quite profound. Attackers from Israel, Iran, Russia, US, North Korea, UK, Europe, and others do this according to the Washington Post [16][17], obviously some much more than others as a whole. DDoS attacks by bots are one of many types of attacks, but one that most of us have encountered, perhaps unwittingly, is the  man-in-the-middle attack, eg., MITM. They use phishing toolkits for spoofing innocent users or bystanders. The immediate handling of these types of attacks and urgency for mitigation, elimination or deterrence is obvious. 

There are methods to avoid and/or determine the existence of a MITM. Traditionally there are eg., Machine Learning methods, ie., the ubiquitous trust-based methods, 2FA, and more new ways, methods, and tools for detection of MITM and more, however they aren't used nor easily available to the general public.
The current status in the field, including the best performing tool is from the basis of our research, with is from the SUNY - Stonybrook paper [1], "Catching Transparent Phish: Analyzing and Detecting MITM  Phishing Toolkits". The name of the extremely efficient tool that they developed is called, "phoca". In brief however, it is discovered that via COTP and of course a shared secret plus a 2FA for security, the APPLICATION layer CAN BE able to help us reveal a MITM if one exists, as opposed to the denial of that concept by Stonybrook, again, knowing that they did a tremendous amount of work on their phoca toolkit. 

What \textit{we} have done is \textit{dispute} this claim by the Stonybrook research group, emphasizing again, that the work that they did is very professional and thorough, these points of relevance will also be exposed and explained in great detail later. 

\noindent
The relevance and importance of COTP will also be explained.\\

\noindent
Source Code Available: https://github.com/allesrebel/cotp\\
The organization of the report is as follows:

\noindent
Abstract\\
Introduction\\
Related Work\\
Design Principles\\
Measurements\\



%---------------------------------------------------------------------------------------------------------------------
\section{Related Work}
[1]Catching Transparent Phish: Analyzing and Detecting MITM Phishing Toolkits (2021, ACM Conference on Computer and Communications Security)
Brian Kondracki, Babak Amin Azad, Oleksii Starov, Nick Nikiforakis.\\

\noindent
[18]Inferring the Presence of Reverse Proxies Through Timing Analysis
Alexander, Daniel R.\\

\noindent
[19]VisualPhishNet: Zero-Day Phishing Website Detection by Visual Similarity
Sahar Abdelnabi, Katharina Krombholz, Mario Fritz\\



%---------------------------------------------------------------------------------------------------------------------
% Still following exactly professor's Canvas example
\section{Design Principles}
MITM Phishing TOOLkitS:\\
Phoca - by our SUNY Stonybrook research group\\
Evilginx, Muraena, and Modlishka [20]\\

% ppt frames 11 , 12 , 13 , 14 . 15
% Paragraph 1: a general overview of the techniques that we used

\noindent	% frame 			15
Design Goals:\\
• Methodology to work on any OS, Windows-10, OSX, Linux distributions,  \indent and/or Android\\
• Ability to work on websites today\\
• Anyone anywhere, should be able to use this to verify a website\\
• Users should have the tool and ability to detect MITMs\\
\textbf{• Validate the server - at the application level} (with a shared secret)\\
• Making MITM significantly harder.\\

\noindent
Figure-1 below shows the typical simplified architecture of a MITM. 
Apparently, the software/interface for the MITM is transparent to the user. What can dilute the transparency are tools like Phoca or Delta-factor, ie., our tool.\\

% frame 			3 = arch of MITM PIX (In professional typesetting, this is called a strut.)
% 		[  : )  pix ] 
  % \usepackage{graphicx}  					Atop - as a preamble
  % \graphicspath{ {~/Desktop/FINALpix/} } 		Atop - as a preamble 
\includegraphics[width=\textwidth]{pix1}
 %\includegraphics[\textwidth]{pix1}  
  
Fig-1 Architecture of a MITM\\ 

% Paragraph 2,3,...: introduction of DT techniques ... [Actually - Inspiration]
\noindent Inspired by an existing movement, a growing trend towards multifactor authentication
identified that most methods still only do knowledge-based or device-based validation.
This is by incorporating existing and standardized One-Time-Pass code technology, with connection-oriented details (eg., COTP)
to produce a method to offer an authentication much stronger than knowledge-based alone. This empowers users to detect the most advanced MITMs today.\\
\indent
The Implementation Details are: design goals, high-level limitations, implementation details (server side and client side), and system integration.\\
\noindent
Standard modules are built into NGINX to extract TLS and TCP connection data. They then pass the details onto a Python WebApp for further validation. In addition, SQLite3 is the user DB,\\



%--------------------------------------------------------------------------------------------------------------------
\section{Measurements}
% TBA
TBA\\
TBA\\



%--------------------------------------------------------------------------------------------------------------------
% Still following exactly professor's Canvas example
\section{Algorithm: \textit{implementation details}}
\noindent
Algorithm fundamentals:\\
Since most authentication methods can be mitigated by applying modern phishing toolkits,
they can be detected through \textit{network traffic}.
There are numerous methods to validate the user through various methods, but \textbf{few server authentication methods}.
We propose reverse 2FA authentication method to \textbf{validate the server from the client perspective}.
We want to validate the link between the server and client by fingerprinting common \textbf{connection} details, and independently verifying shared secrets.\\ 

% Paragraph 6: introduction of RF techniques 		[Actually Delta factor/COTP]
\noindent
Algorithm design:\\
A synopsis on \textbf{COTP} is created by:\\
\noindent
A high e, (entropy) shared secret,\\
extraction of shared low level network features,\\
creating a shared time frame,\\ 
creating a composite secret from from the features and high-e secret,\\
generating a hash from the composite secret,\\
% Compress/Truncate hash, Yielding a Connection_based OTP
having the \textit{server} send the generated COTP, and\\
comparing the client and server COTPs. At this point, the connection is accepted or declined.\\

%----------------------------------- <> % <>-----------------------------------
\noindent
\textbf{Code} type \textbf{"pseudocode"} (because of simplicity):\\
\noindent
Used existing HTTP Server code built into Py\\
\noindent
Developed a simple \textbf{COTP} algorithm from reference\\

\noindent
FOR INPUT:\\
\noindent
import hmac, base64, struct, hashlib, time\\

\noindent
def get\_hotp\_token(secret, msg):\\
\indent key = base64.b32decode(secret, true)\\

\# secret -\textgreater just unsigned bytes\\
\indent msg\_array = bytearray()\\
\indent msg\_array.extend(map(ord, msg))\\

\# RFC says to use specific bytes\\
\indent h = hmac.new(key, msg\_array, hashlib.sha1).digest()\\
\indent o = h[19] \& 15\\

\# Generate a hash using HMAC SHA1\\
\indent \# Grab the first 3 bytes after 20th, undo endiness\\
\indent key\_byte = struct.unpack("\textgreater l", h[o:o+4])[0]\\
\indent \# Convert key byte into a code\\
\indent htop =  (key\_byte \& 0x7fffffff) \% 1000000\\
\indent return htop\\

\noindent
EXAMPLE INPUT (from NGINX)\\
\noindent
tcp\_rtt=20380\\
\noindent
tls\_proto='TLSv1.3'\\
\noindent
tls\_csuite='TLS\_AES\_128\_GCM\_SHA256'\\
\noindent
secret='TESTTESTTEST===' \# 16 chars\\
%-----------
%-----------
%-----------

\noindent
FOR OUTPUT:\\
\noindent
def get\_cotp(secret):\\
\indent \# Gather 30sec timeframe from time - now!\\
\indent time\_Frame = (int(time.time())//30)\\
\indent \# Gather/extract out cipher suite + tcp RTT\\
\indent cipher\_suite = tls\_csuite 		         \# See above\\
\indent protocol\_version = tls\_proto		         \# See above\\
\indent tcp\_rt\t\_ms = (tcp\_rtt//1000)		         \# 34microsec\\     

\# "Stitching" everything together;\\
\indent
msg = str(time\_Frame) + str(cipher\_suite) \indent + str(protocol\_version) + \indent str(tcp\_rtt\_ms)\\

\# Ensuring to give the same OPT for 30sec\\
\indent cotp = str(get\_hotp\_token(secret, msg))\\

\# Adding 0 in the beginning till OTP has 6 digits\\
\indent while len(cotp) !=6:\\
 \indent \indent cotp += '0'\\

\indent \indent return cotp\\

\noindent
EXAMPLE OUTPUT\\
 \noindent
 COTP (30 second windows)\\
 \textbf{321009}\\
 \textbf{500028}\\ 
%----------------------------------- <> % <>-----------------------------------



%--------------------------------------------------------------------------------------------------------------------
\section{Experiment}				% Still following exactly professor's Canvas example
\noindent
Setup:\\
setup\\

\noindent
Details:\\
details\\

\noindent
Results:\\
results\\
figures and observations\\
analysis of the results\\



%------------------------------------------------------------------------------%---------------------------------------
%------------------------------------------------------------------------------%---------------------------------------
\section{CITATIONS \normalfont{(Chicago style):}} 	% Same as "References" as per prof

\noindent
[1]Catching transparent Phish: Analyzing and Detecting MITM Phishing Toolkits\\
\noindent 
Brian Kondracki, Babak Amin Azad, Oleskii Starov, Nick Nikiforakis\\
Stonybrook University and Palo Alto Networks\\ 
\noindent
ACM ISBN-978-1-4503-8454-4/21/11 Nov 2021\\

\noindent
[2]https://datatracker.ietf.org/doc/html/rfc4226 - Creating a HOTP\\

\noindent
[3]https://datatracker.ietf.org/doc/html/rfc6238 - Create a TOTP\\ 

\noindent
[4]https://datatracker.ietf.org/doc/html/rfc4086 - High Entropy Secrets\\

\noindent
[5]https://datatracker.ietf.org/doc/html/rfc2104 - HMAC Algorithm\\

\noindent
[6]Test Website Live: http://allesrebel.com\\ 

\noindent
Reference implementations provided by papers + standardized code

\noindent
[7]TOTP - https://datatracker.ietf.org/doc/html/rfc6238

\noindent
[8]HOTP - https://datatracker.ietf.org/doc/html/rfc4226 

\noindent
[9]COTP implementation - https://github.com/TomEphraimPerez/CS5540-final/tree/webserver /nginx\\

\noindent
[10]Collecting TCP Timings via Performance API\\
https://developer.mozilla.org/en-US/docs/Web/API/Performance\\

\noindent
[11]Collecting TLS Information via Security Info from Extensions JS API\\
https://developer.mozilla.org/en-US/docs/Mozilla/Add-ons/WebExtensions/API/webRequest /SecurityInfo\\

\noindent
[12]Store Secret via Browser Extension Storage API\\
https://developer.mozilla.org/en-US/docs/Web/API/Storage\\

\noindent
[13]Crypto APIs + Time APIs to generate COTP\\
https://developer.mozilla.org/en-US/docs/Web/API/Crypto\\


\noindent
[14]JA3, https://github.com/salesforce/ja3\\

\noindent
[15]https://coveryourtracks.eff.org/, Proceedings of the Privacy Enhancing Technologies Symposium\\

\noindent
[16]https://www.washingtonpost.com/technology/2021/07/09/how-ransomware-attack-works/\\

\noindent
[17]https://www.washingtonpost.com/technology/2022/03/07/russia-belarus-conducted-widespread-phishing-campaigns-ukraine-google-says/\\

\noindent
[18]Inferring the Presence of Reverse Proxies Through Timing Analysis Alexander, Daniel R.\\
https://apps.dtic.mil/sti/citations/ADA632473 US naval Post-Graduate School - Cybersecurity\\

\noindent
[19]VisualPhishNet: Zero-Day Phishing Website Detection by Visual Similarity Sahar Abdelnabi, Katharina Krombholz, Mario Fritz\\
https://dl.acm.org/doi/10.1145/3372297.3417233\\

\noindent
[20][Evilginx, Muraena, and Modlishka]\\
https://cybernews.com/security/researchers-find-more-than-1200-phishing-toolkits-across-the-web/\\

\noindent
[-]Public Internet Standards (Various Authors)\\

%---------------------------------------------------------------------------------------------------------------------
\section{Conclusion}
conclusions\\
more\\



\end{document}


